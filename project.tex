% Options for packages loaded elsewhere
% Options for packages loaded elsewhere
\PassOptionsToPackage{unicode}{hyperref}
\PassOptionsToPackage{hyphens}{url}
\PassOptionsToPackage{dvipsnames,svgnames,x11names}{xcolor}
%
\documentclass[
  letterpaper,
  DIV=11,
  numbers=noendperiod]{scrartcl}
\usepackage{xcolor}
\usepackage{amsmath,amssymb}
\setcounter{secnumdepth}{-\maxdimen} % remove section numbering
\usepackage{iftex}
\ifPDFTeX
  \usepackage[T1]{fontenc}
  \usepackage[utf8]{inputenc}
  \usepackage{textcomp} % provide euro and other symbols
\else % if luatex or xetex
  \usepackage{unicode-math} % this also loads fontspec
  \defaultfontfeatures{Scale=MatchLowercase}
  \defaultfontfeatures[\rmfamily]{Ligatures=TeX,Scale=1}
\fi
\usepackage{lmodern}
\ifPDFTeX\else
  % xetex/luatex font selection
\fi
% Use upquote if available, for straight quotes in verbatim environments
\IfFileExists{upquote.sty}{\usepackage{upquote}}{}
\IfFileExists{microtype.sty}{% use microtype if available
  \usepackage[]{microtype}
  \UseMicrotypeSet[protrusion]{basicmath} % disable protrusion for tt fonts
}{}
\makeatletter
\@ifundefined{KOMAClassName}{% if non-KOMA class
  \IfFileExists{parskip.sty}{%
    \usepackage{parskip}
  }{% else
    \setlength{\parindent}{0pt}
    \setlength{\parskip}{6pt plus 2pt minus 1pt}}
}{% if KOMA class
  \KOMAoptions{parskip=half}}
\makeatother
% Make \paragraph and \subparagraph free-standing
\makeatletter
\ifx\paragraph\undefined\else
  \let\oldparagraph\paragraph
  \renewcommand{\paragraph}{
    \@ifstar
      \xxxParagraphStar
      \xxxParagraphNoStar
  }
  \newcommand{\xxxParagraphStar}[1]{\oldparagraph*{#1}\mbox{}}
  \newcommand{\xxxParagraphNoStar}[1]{\oldparagraph{#1}\mbox{}}
\fi
\ifx\subparagraph\undefined\else
  \let\oldsubparagraph\subparagraph
  \renewcommand{\subparagraph}{
    \@ifstar
      \xxxSubParagraphStar
      \xxxSubParagraphNoStar
  }
  \newcommand{\xxxSubParagraphStar}[1]{\oldsubparagraph*{#1}\mbox{}}
  \newcommand{\xxxSubParagraphNoStar}[1]{\oldsubparagraph{#1}\mbox{}}
\fi
\makeatother


\usepackage{longtable,booktabs,array}
\usepackage{calc} % for calculating minipage widths
% Correct order of tables after \paragraph or \subparagraph
\usepackage{etoolbox}
\makeatletter
\patchcmd\longtable{\par}{\if@noskipsec\mbox{}\fi\par}{}{}
\makeatother
% Allow footnotes in longtable head/foot
\IfFileExists{footnotehyper.sty}{\usepackage{footnotehyper}}{\usepackage{footnote}}
\makesavenoteenv{longtable}
\usepackage{graphicx}
\makeatletter
\newsavebox\pandoc@box
\newcommand*\pandocbounded[1]{% scales image to fit in text height/width
  \sbox\pandoc@box{#1}%
  \Gscale@div\@tempa{\textheight}{\dimexpr\ht\pandoc@box+\dp\pandoc@box\relax}%
  \Gscale@div\@tempb{\linewidth}{\wd\pandoc@box}%
  \ifdim\@tempb\p@<\@tempa\p@\let\@tempa\@tempb\fi% select the smaller of both
  \ifdim\@tempa\p@<\p@\scalebox{\@tempa}{\usebox\pandoc@box}%
  \else\usebox{\pandoc@box}%
  \fi%
}
% Set default figure placement to htbp
\def\fps@figure{htbp}
\makeatother





\setlength{\emergencystretch}{3em} % prevent overfull lines

\providecommand{\tightlist}{%
  \setlength{\itemsep}{0pt}\setlength{\parskip}{0pt}}



 


\usepackage{caption}
\captionsetup{
labelfont=bf,
width=0.9\linewidth
}
\KOMAoption{captions}{tableheading}
\makeatletter
\@ifpackageloaded{caption}{}{\usepackage{caption}}
\AtBeginDocument{%
\ifdefined\contentsname
  \renewcommand*\contentsname{Table of contents}
\else
  \newcommand\contentsname{Table of contents}
\fi
\ifdefined\listfigurename
  \renewcommand*\listfigurename{List of Figures}
\else
  \newcommand\listfigurename{List of Figures}
\fi
\ifdefined\listtablename
  \renewcommand*\listtablename{List of Tables}
\else
  \newcommand\listtablename{List of Tables}
\fi
\ifdefined\figurename
  \renewcommand*\figurename{Figure}
\else
  \newcommand\figurename{Figure}
\fi
\ifdefined\tablename
  \renewcommand*\tablename{Table}
\else
  \newcommand\tablename{Table}
\fi
}
\@ifpackageloaded{float}{}{\usepackage{float}}
\floatstyle{ruled}
\@ifundefined{c@chapter}{\newfloat{codelisting}{h}{lop}}{\newfloat{codelisting}{h}{lop}[chapter]}
\floatname{codelisting}{Listing}
\newcommand*\listoflistings{\listof{codelisting}{List of Listings}}
\makeatother
\makeatletter
\makeatother
\makeatletter
\@ifpackageloaded{caption}{}{\usepackage{caption}}
\@ifpackageloaded{subcaption}{}{\usepackage{subcaption}}
\makeatother
\usepackage{bookmark}
\IfFileExists{xurl.sty}{\usepackage{xurl}}{} % add URL line breaks if available
\urlstyle{same}
\hypersetup{
  pdftitle={Project},
  pdfauthor={Devendra Singh},
  colorlinks=true,
  linkcolor={blue},
  filecolor={Maroon},
  citecolor={Blue},
  urlcolor={Blue},
  pdfcreator={LaTeX via pandoc}}


\title{Project}
\author{Devendra Singh}
\date{}
\begin{document}
\maketitle

\renewcommand*\contentsname{Table of contents}
{
\hypersetup{linkcolor=}
\setcounter{tocdepth}{3}
\tableofcontents
}
\listoffigures

\pagebreak

\subsubsection{Introduction}\label{introduction}

In this project we are going to simulate the Swiss Mountain glacier,
Arolla Glacier. In order to do that we are going to model the ice
velocity in a 2D cross section represented by the PDE called First order
model. These equations are non-linear as the viscosity depends on
horizontal velocity(Equation~\ref{eq-5}) making it an non- Newtonian
fluid, represented as follows:

\begin{quote}
FO model:
\end{quote}

\begin{equation}\phantomsection\label{eq-main}{
2\partial_x(\eta\partial_x u_x)+\frac{1}{2}\partial_z(\eta\partial_z u_x)=\rho g\partial_x h
}\end{equation}

\begin{equation}\phantomsection\label{eq-2}{
\partial_x u_z = - \partial_x u_x
}\end{equation}

\begin{equation}\phantomsection\label{eq-3}{
\partial_t h + u_x|_{z=s} \partial_x h = u_z
}\end{equation}

\begin{quote}
Boundary Conditions:
\end{quote}

\begin{quote}
\begin{quote}
At the bed surface(Dirichlet's): \(u_x=0\)
\end{quote}
\end{quote}

\begin{quote}
\begin{quote}
At the ice/atmosphere interface(Neumann):
\end{quote}
\end{quote}

\begin{equation}\phantomsection\label{eq-8}{
\eta(2\partial_{x} u_x \partial_{x} h -\frac{1}{2}\partial_{z}  u_x) =  0
}\end{equation}

\begin{quote}
Viscosity dependence on horizontal velocity:
\end{quote}

\begin{equation}\phantomsection\label{eq-5}{
\eta= A^{-1/3} \left( (\partial_xu_x)^2+\frac{1}{4}(\partial_zu_x)^2\right)^{-1/3}
}\end{equation}

\pagebreak

\subsubsection{1. Order of equation :}\label{order-of-equation}

The Equation~\ref{eq-main}, is \textbf{second order} for horizontal
velocity.

\subsubsection{2. Type of equation :}\label{type-of-equation}

For constant viscosity(\(\eta\)), the PDE for horizontal velocity
Equation~\ref{eq-main} can be written as,

\[
\eta(2\partial_{xx} +\frac{1}{2}\partial_{zz} ) u_x =  \rho g\partial_x h
\]

When we put values in the discriminat(\(B^2-4AC\)), we get it as
negative since the term representing the coefficient B is absent and A,
C both are positive, suggesting that its an \textbf{elliptic} PDE of
second order.

\subsubsection{3. FEM formulism :}\label{fem-formulism}

To solve Equation~\ref{eq-main} using FEM, we first need to convert this
PDE into weak(Variational) form. For that, we define a function space
\(V_h\) over our mesh \(\Omega\). In this function space we find
\(u_x \in V_h\) such that \(u_x=0\) on the bed(Homogeneous Dirichlet
BC).

\[
V_h = \{ v \in H^1(\Omega) : u|_{\Gamma_{bed}} = 0 \}
\]

The weak form to find \(u \in V_h\) for all \(v \in V_h\) has ,

\begin{quote}
the bilinear form
\end{quote}

\begin{equation}\phantomsection\label{eq-6}{a(u, v) = \int_{\Omega} \left( 2\eta \partial_x u \partial_x v + \frac{1}{2}\eta \partial_z u \partial_z v \right) \, d\Omega}\end{equation}

\begin{quote}
and the linear form,
\end{quote}

\begin{equation}\phantomsection\label{eq-7}{
L(v) = -\int_{\Omega} \rho g \partial_x h v \, d\Omega 
}\end{equation}

The boundary condition:

\begin{quote}
At the bed surface(\(\Gamma_{bed}\)): \(u_x=0\)
\end{quote}

\begin{quote}
At the ice/atmosphere interface(\(\Gamma_{surface}\)):
\end{quote}

\begin{equation}\phantomsection\label{eq-8}{
\eta(2\partial_{x} v_x \partial_{x} h -\frac{1}{2}\partial_{z}  v_x) =  0
}\end{equation}

The set of equations Equation~\ref{eq-6}, Equation~\ref{eq-7},
Equation~\ref{eq-8} represent the well posed and complete problem which
can be solved linearly in FeniCS.

\subsubsection{\texorpdfstring{4. Linear version of
Equation~\ref{eq-main}}{4. Linear version of Equation~}}\label{linear-version-of-eq-main}

To achieve linearity, we have set the viscosity as 1.




\end{document}
